%% This is an example first chapter.  You should put chapter/appendix that you
%% write into a separate file, and add a line \include{yourfilename} to
%% main.tex, where `yourfilename.tex' is the name of the chapter/appendix file.
%% You can process specific files by typing their names in at the 
%% \files=
%% prompt when you run the file main.tex through LaTeX.
\chapter{Objectives}
As indicated by the analysis of existing CAD and computational design tools, many options exist to support novice entry into computer science and new tools are emerging that provide different ways of engaging in CAD and digital fabrication. At this point however, is a lack of tools, which attempt to bridge accessible programing and novel forms of digital fabrication. The primary objective of this thesis is to research the relationship between textual programing language and visual design, and physical construction by developing tools that allow for casual, craft-oriented applications of digital fabrication and computational design.

	\section{Functional Properties of Algorithmic Crafting Tools}
	Based prior CAD and computational design tools I hypothesized that the following functional properties are necessary for an effective algorithmic craft software:
	\begin{itemize}
\item \textbf{Emphasis on computational design:} Programing should be the chief method of generating and manipulating designs, and this focus should be reflected in the interface of the software. 
\item \textbf{ Novice-oriented programing syntax:} The programing syntax and application programing interface (API) should be designed to accommodate individuals with no prior design experience, and should be limited to methods and structures relevant to the task of design. 
\item \textbf{Support of sophisticated design techniques:} Although the programatic aspects of the program should seek to cater to first time programmers, the tool should work for a range of design expertise and objectives:  from amateurs with an expressed interest in design and aesthetics, to those with prior design experience. 
\item \textbf {Design methods that facilitate digital fabrication:} The software should include programing and drawing methods that allow for the production of designs that are suitable for digital fabrication and support for exporting to relevant file formats. 
\item \textbf{ Prioritization of visualization and simulation:} Users  should be given ample visual feedback to inform their programming decisions.
\item \textbf{ Simple workflow from software to fabrication:} The transition from the design tool to the fabrication device should require as few intermediary steps as possible.
\item \textbf{ Free and open-source:} The software should be freely available, and able to function on multiple platforms with low requirements for computational processing power to afford high levels of access to casual users. If possible, the software should also be open source, in order to encourage the proliferation of additional novice oriented tools that can be developed for the specific needs of distinct user groups.
\item \textbf{Domain specificity:} In order to support first-time users, the tool should be constrained to a the design of a particular set of end products. Or, if the software is general purpose, it should be packaged with a set of well documented example designs and projects that clearly demonstrate its key applications. 
\item \textbf{Fabrication and craft-specific documentation:} In addition to the documentation of the application and programing language, the fabrication and crafting techniques intended for use with the software should be well-documented. This documentation should include details on suitable materials, fabrication machine access and settings, and tutorials on the craft components of example projects.
\end{itemize}

\section{Evaluation Criteria}
In addition to these functional properties, I generated a set of evaluation criteria for any prospective algorithmic crafting software. A successful tool should produce the following results:
\begin{itemize}

\item \textbf{Allow users to successfully create physical artifacts:} The artifacts themselves should be durable and useful.
\item \textbf{Afford a wide degree of variation in design and expression:} The personal aesthetic preferences of the creator should be apparent in the resultant artifact.
\item \textbf{Enable people to understand the functionality and utility of the programs they write} Individuals should emerge from the process with a general understanding of some of the key components of computer programing, with an ability to articulate how these components function in their design. 
\item \textbf{Allow users to create objects and designs they would have difficulty generating with conventional techniques} The tool should support the affordances of computational design expressed in section \ref{sec:computational_design}, specifically precision, visual complexity, generativity and stylistic abstraction.
\item \textbf{Engender in users a positive, enjoyable experience:} Use of the tool and subsequent crafting activities should be pleasurable.
\item \textbf{Foster a sense of confidence:} After working with the tool, people should have increased confidence in their ability to successfully program, design, and use digital fabrication tools. 
\end{itemize}

\section{Design Tools and Evaluation Methodology}
Over the course of my thesis, I developed and tested three software tools to support algorithmic craft. Codeable Objects is a domain specific programing library for the design and production of lamps. Soft Objects is an expanded version of Codeable Objects aimed at computational fashion design. DressCode is a general-purpose integrated programing and visual design environment. \todo{repetitive from introduction?} Each tool was evaluated during one or more workshops with designers, artists, programmers and young people. I  documented each workshop through pre and post surveys, interviews, and photographs of student projects.  The surveys were aimed at understanding participants� previous experience in programming and design, their interest in and attitudes toward programming and design (before and after the workshops), and their engagement in and enjoyment of the workshops.

Pre-surveys were administered at the start of the workshops and focused on participants� previous experience and attitudes. They also asked students to describe their opinions about how programing and craft could be combined, and how they felt programming could extend or limit creativity. Post-surveys were administered at the termination of the workshops and contained attitudinal questions that were matched to the pre-surveys. In addition, post surveys contained a range of written questions asking the participants to describe their opinion of the success of their projects and their experience using Codeable Objects, Soft Objects or DressCode respectively.  

Individual interviews were conducted with the participants in the Soft Objects workshop, and the DressCode workshop. These interviews lasted an average of 15-30 minutes and were audio recorded and transcribed. During the interviews, the participants were asked to describe their experience in the workshop and talk about the process of conceptualizing, designing and producing their garments. They were asked to describe what they enjoyed, what was difficult for them, and what they felt they had learned through this process. A listing of the interview questions is available in appendix \ref{app:interviews}. Survey and verbal interview responses and project outcomes were then analyzed to determine if the essential qualities outlined in the requirements section (above) were achieved.  We also used this information to identify recurring and prominent themes in participants� experiences. In the following  three chapters, I detail the development, feature set and evaluation of each design tool.