%% This is an example first chapter.  You should put chapter/appendix that you
%% write into a separate file, and add a line \include{yourfilename} to
%% main.tex, where `yourfilename.tex' is the name of the chapter/appendix file.
%% You can process specific files by typing their names in at the 
%% \files=
%% prompt when you run the file main.tex through LaTeX.
\chapter{Objectives and Evaluation Criteria}
As indicated by the analysis of existing CAD and computational design tools, many options exist to support novice entry into computer science and new tools are emerging that provide different ways of engaging in CAD and digital fabrication. At this point however, there are a lack of tools, which attempt to bridge accessible programing and novel forms of digital fabrication in relevant and engaging contexts for amateur programmers. Digital fabrication and computational design are two highly compatible domains with great creative potential. When combined with the values, practices, and aesthetics of craft, these shared practices offer new forms of expression and making. The objective of this thesis was to foster engagement in algorithmic craft by researching the relationship between programing, visual design, and physical construction. I conducted this research through the development of software tools that facilitated the combined practice of digital fabrication, computational design, and craft.  In creating these tools, I emphasized computation by positioning programing as the primary method of generating and manipulating designs. In doing so however, I attempted to retain a high degree of accessibility and intelligibility for new programmers. Upon the basis of this general objective, I generated a set of evaluation criteria for any prospective algorithmic crafting software. A successful tool should produce the following results:
\begin{itemize}

\item \textbf{Allow users to successfully create physical artifacts:} The artifacts themselves should be durable and useful.
\item \textbf{Afford a wide degree of variation in design and expression:} The personal aesthetic preferences of the creator should be apparent in the resultant artifact.
\item \textbf{Enable people to understand the functionality and utility of the programs they write:} Individuals should emerge from the process with a general understanding of some of the key components of computer programing, with an ability to articulate how these components function in their design. 
\item \textbf{Allow users to create objects and designs they would have difficulty generating with conventional techniques:} The tool should support the affordances of computational design, specifically precision, visual complexity, generativity and stylistic abstraction, as well as enabling people to take advantages of the properties of digital fabrication including  manufacturing speed, precision, and physical complexity. 
\item \textbf{Engender in users a positive, enjoyable experience:} Use of the tool and subsequent crafting activities should be pleasurable and intellectually engaging.
\item \textbf{Foster a sense of confidence:} After working with the tool, people should have increased confidence in their ability to successfully program, design, and use digital fabrication tools. 
\end{itemize}

\section{Design Tools and Evaluation Methodology}
Over the course of my thesis, I developed and tested three software tools to support algorithmic craft. Codeable Objects is a domain specific programing library for the design and production of lamps. Soft Objects is an expanded version of Codeable Objects aimed at computational fashion design. DressCode is a general-purpose integrated programing and visual design environment. Each tool was evaluated during one or more workshops with designers, artists, programmers and young people. I  documented each workshop through pre and post surveys, interviews, and photographs of student projects.  The surveys were aimed at understanding participants' previous experience in craft, programming and design, their interest in and attitudes toward craft, digital fabrication, and computation (before and after the workshops), and their engagement in and enjoyment of the workshops.

Pre-surveys were administered at the start of the workshops and focused on participants� previous experience and attitudes. They also asked students to describe their opinions about how programing and craft could be combined, and how they felt programming could extend or limit creativity. Post-surveys were administered at the termination of the workshops and contained attitudinal questions that were matched to the pre-surveys. In addition, post surveys contained a range of written questions asking the participants to describe their opinion of the success of their projects and their experience using Codeable Objects, Soft Objects or DressCode respectively.  

Individual interviews were conducted with the participants in the Soft Objects workshop, and the DressCode workshop. These interviews lasted an average of 15-30 minutes and were audio recorded and transcribed. During the interviews, the participants were asked to describe their experience in the workshop and talk about the process of conceptualizing, designing and producing their artifacts. They were asked to describe what they enjoyed, what was difficult for them, and what they felt they had learned through this process. Survey and verbal interview responses and project outcomes were then analyzed to determine if the essential qualities outlined in the evaluation criteria were achieved.  We also used this information to identify recurring and prominent themes in participants� experiences. In the following  three chapters, I detail the development, feature set and evaluation of each design tool.