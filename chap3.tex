%% This is an example first chapter.  You should put chapter/appendix that you
%% write into a separate file, and add a line \include{yourfilename} to
%% main.tex, where `yourfilename.tex' is the name of the chapter/appendix file.
%% You can process specific files by typing their names in at the 
%% \files=
%% prompt when you run the file main.tex through LaTeX.
\chapter{Related Tools and Research}
There are numerous forms of CAD software and programing environments. Within the realm of computational design and digital fabrication, there are 4 primary categories of existing tools  that directly relate to my study of computational design and digital fabrication: professional computational-design tools, entry-level programing environments, and novice-oriented computer-aided-design (CAD) tools. While certain qualities are shared between these categories, several key distinctions exist between each group of tools.  

\section{Professional Computational Design Tools}
A couple of forms of professional computational-design tools exist. Foremost, many popular graphic-user-interface (GUI) CAD applications include a feature that allows the user to automate certain elements of the program through scripting or programing. For example, in Adobe software like Photoshop and Illustrator, it is possible to write JavaScript-based programs to automate various application procedures. Similarly, 3D modeling tools such as Maya and Blender feature the ability to script behaviors in languages that are syntactically similar to Perl and Python respectively. This scripting is usually omitted from the primary menus and interfaces of the applications that feature it. 
There are also professional tools that are explicitly developed for computational design. The most prominent example is Grasshopper, a third-party add-on for the Rhinoceros 3D modeling tool. Grasshopper is a data-flow programing environment that lets users combine a variety of modules and blocks to create and adjust 3D models in Rhino. A textual coding module is also available and allows users to integrate C\# scripts using the Rhino API into their patch, although the user must have an understanding of the basic principles of programing in order to effectively use this functionality. 

DesignScript, a more recent computational design tool, developed by Autodesk, is a domain specific text-based programing environment and language that contains methods to generate and manipulate geometric models that are compatible with existing Autodesk applications. DesignScript itself functions as an add on to the Autodesk AutoCad software. DesignScript is intended for use by experienced designers  and 3d modelers who posses a range of programming expertise. The language syntax is based on C\#, however it features the ability to operate in both associative and imperative paradigms, in an effort to support a pedagogical transition between basic and more complex forms of computational design \cite{DesignScript}. 

Lastly, OpenSCAD is a script-based constructive solid geometry modeling tool developed specifically for CAD applications. OpenSCAD contains a custom programing language in which the user can create descriptions of 3d models in a textual format, and display them by compiling the script. This scripting behavior provides the user with precise control over the modeling process and enables the creation of designs that are defined by configurable parameters, however this control comes at the cost of requiring the user to be familiar with textual programing an scripting. In fact, OpenSCAD is explicitly developed for programmers and relies on textual input exclusively as the mechanism for design, In addition, unlike the prior tools mentioned, OpenSCAD is both free and open source, and many variations and derivatives of it exist \cite{OpenScad}.

In the context of digital fabrication, one of the most important elements of these professional tools is their ability to import and export a wide variety of file formats, thus facilitating the transitions between a digital design and the required file type for a specific fabrication tool. Despite their power, and due to their high cost and complex feature set, these professional tools are extremely difficult for amateurs to access and use. It is also important to note that with the exception of OpenSCAD, the examples listed are only available as plugins or add-ons  or are developed to supplement an existing graphical tool, rather than serve as the primary method of design. In some cases this status as a form secondary functionality adds a set of practical barriers to independent use. The scripting tools in illustrator and photoshop are difficult to locate, Grasshopper only functions on Windows versions of Rhino, and Design Script requires the prior purchase of AutoCAD to operate. Although these practical barriers can be overcome, their existence often prevents less experienced users from gaining access. In addition, the positioning of computational functionality as secondary to the primary method of design points to a larger ideological classification of these forms of design as a specialized and exclusive, rather than a primary method of design. 

\section{Entry-level CAD Tools}
A subset of CAD tools have also been created that are designed to be more accessible to a wider range of people. These tools provide an option for individuals who lack the experience and access to professional level tools, however they also provide an opportunity for more casual participation in CAD. SketchUp is a 3d modeling tool developed by Google to enable easier forms of 3D modeling. Although SketchUp was not explicitly created to allow people to design for CNC and digital fabrication, several 3rd-party add ons exist that allow users to export designs to file formats that are compatible with a variety of fabrication machine \cite{sketchup}. TinkerCad is another 3d modeling tool designed for entry level users.  As opposed to SketchUp, TinkerCad is explicitly developed to assist in designing for 3d printers and has built in functionality to allow users to export their designs to the .stl format which is compatible with 3D printing \cite{tinkercad}. AutoDesk has also produced several entry level 3d-modeling applications as a part of their 123D series. Many of these applications are designed to interface with digital fabrication, including 123D Make which allows users to convert stock or uploaded 3d models into a series of flat parts which can be fabricated on 2-axis machines like laser cutters, and 123D Creature, which enables users to design a variety of creatures from a set of basic parts and then order a 3d printed model of their finished creature \cite{123D}.  AutoDesk Research has also developed MeshMixer, an application for the intuitive merging and manipulating high resolution triangle meshes. MeshMixer was released to the public and has since become a popular 3d design tool for hobbyist 3D printer users. 
All of the entry level tools listed above vary in their specific approach to creating more accessible forms of CAD. In general they feature a trade off between limited functionality and power, in favor of a simplified tool set and an easier learning curve. Despite these restrictions, it is possible to use these entry level tools to develop highly  complex and sophisticated models \todo{show example image of mesh mixer model}. A more serious limitation of these tools is their ephemerality. Because entry level CAD tools are often free, and more frequently web based applications, it is common for them to suddenly become unavailable or no longer supported by the company that produces them. Tinkercad serves as a recent example of this wherein the parent company decided to transition to focusing on professional-level CAD tools and as a result, closed down the Tinkercad website and cut off access to the application \todo{footnote about tinkercad recently being acquired by autodesk}. 
Several of these entry-level tools feature some form of scripting or programing functionality. A plugin for Sketchup allows users to automate certain actions by using the Ruby-based SketchUp API. TinkerCad allows users to create Shape Scripts, which are parametric models defined by javascript code. MeshMixer has an C++ API which is not yet publicly available, but is provided to interested parties upon request. While these computational tools suggest compelling possibilities, similar to the professional level tools listed above, they are positioned as secondary ways of interacting, and are much less deliberate than the primary features of the application.

\section{Learning-Oriented programming tools}
In addition to entry level CAD tools, a number of tools and applications have been created to introduce inexperienced programmers to the realm of computer science. 
Logo, a computational drawing program, serves as the seminal novice programming language founded on principles of constructionism and embodiment [9]. The Scratch visual programming language is a notable successor to Logo, and allows users to create interactive projects by combining command blocks rather than writing textual code [11]. Alice is another programming environment that relies on visual programming, but is targeted towards an older user group than Scratch [3]. Turtle Art [16] and Design Blocks [2] are two visual programming languages inspired by Logo that are designed specifically for visual composition. Processing is a text-based programming environment designed for easy learning, and directed toward artists, designers, and inexperienced programmers [10].
Logo, Turtle Art, Design Blocks, and Processing facilitate computational drawing and, therefore, can be viewed as computational-design environments. There remains a gap, however, between novice-oriented programing environments and the novice-oriented CAD tools. In direct contrast to the novice oriented CAD tools described in the preceding section, although learning oriented programing tools can provide an excellent platform for generating digital computational design work, they often lack explicit features for generating and exporting designs that are compatible digital fabrication. It is possible to create work-arounds to this. For example in processing, users can download and install community-created libraries that allow for .stl, .dxf and .pdf export, enabling a sub group of users to use processing as a design tool for 3d printing, and laser cutting. The independent development of export functionality for tools like Processing demonstrates that there is significant interest in combining computational design and digital fabrication. If we wish to open this space for entry level practitioners however, we must design tools that exhibit the tools and techniques for computational design for fabrication as their primary functionality.


\section{Novel Fabrication and CAD tools}
In addition to these tools, there are a number of research projects involving novel forms of fabrication and software tools.  Most of these tools are highly domain specific, however they provide insight into the development of more general approaches for computational design and digital fabrication. 
Midas
Sketch Chair
FlatCad
Natalie's spirograph


/Garment-Creation CAD and Fabrication tools
Sensitive Couture
parsing patterns into 3d garments
Sketch Based Garment Design
Art quilt?