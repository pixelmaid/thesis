%% This is an example first chapter.  You should put chapter/appendix that you
%% write into a separate file, and add a line \include{yourfilename} to
%% main.tex, where `yourfilename.tex' is the name of the chapter/appendix file.
%% You can process specific files by typing their names in at the 
%% \files=
%% prompt when you run the file main.tex through LaTeX.
\chapter{Design Tools}
\todo{introduction to design tools}

\section{Codeable Objects}
Codeable Objects is computational design tool that allowed people to design a laser cut lamp. The choice of a lamp allowed for a relatively broad design space wherein aesthetics were a primary consideration, while still retaining the qualities of functionality and utility in the finished product.  Lamps possess an established function, but offer a great deal of flexibility and personal freedom in the aesthetics and form. In addition, there is an established history of creating DIY lamps via digital fabrication. The Instructables community tutorial website has an entire section devoted to DIY lamps, and many examples of patterns that use a laser cutter for fabrication. 
\begin{center}
\begin{figure}[h!]
\includegraphics[width=6.5in]{images/instructables_lamps.png}
\caption{a selection of laser cut lamps from Instructables}
\end{figure}
\end{center}
\subsection{Motivation}
One of the restrictions of many of these examples is that they require the person making the lamp to directly emulate the design provided by the creator of the tutorial. If the person wishes to deviate from the original design, they  need to use a CAD tool like Adobe Illustrator or Solid Works\cite{instructables_lamp_1}. As mentioned in Section \ref{sec:professional_computational_design_tools}, professional CAD tools like Solid Works are often difficult to access and use for casual practitioners. In addition, during my personal experience in using a non-parametric tool like illustrator to design, I often found I had to resort to fabricating numerous sample pieces of in order to ensure the joints and form would function correctly in the final piece. If I made a mistake, or decided I wanted to modify the design, I lost time and materials in the fabricating process, and had to endure the tedious process of adjusting correcting each individual part. 
\begin{center}
\begin{figure}[h!]
\includegraphics[width=6.5in]{images/solidworks_lamp.png}
\caption{Instructables lamp tutorial with SolidWorks design process}
\end{figure}
\end{center}
One of the most frequent applications of a laser cutter is to create 3D forms by assembling 2D press fit pieces in a frame-like structure. I found that when creating 3D forms that were curved, it was extremely challenging in traditional 2D CAD software to correctly size and design parts which would fit the faces of the form. This was particularly relevant to lamp design, wherein it was necessary to create shades to diffuse the light. The shades also provided an excellent space for incorporating styles and patterns into the lamp. The combined tasks of simplified design and customization, parametric manipulation, and the calculation and conversion of a 3D form to 2D parts indicated that computational design would be a good match for the task of designing and fabricating a laser cut lamp. 
	
\subsection{Tool Description and workflow}
The objective of the first version of Codeable Objects was simple: to create a tool that allowed to design a custom lamp by describing the form and the pattern of the shades, which they could then fabricate and assemble. The lamp itself was comprised of 4 basic parts, a wooden press fit frame, a set of vellum pieces that fit over the frame to act as a shade,a set of cardstock pieces with a pattern that fit over the shades, and a commercial made light fixture that fit into the frame (see figure: \ref{fig:lamp_parts}.)
\begin{center}
\begin{figure}[h!]
\includegraphics[width=5.337in]{images/parts.png}
\caption{the individual parts of a lamp}
\label{fig:lamp_parts}
\end{figure}
\end{center}
Codeable Objects was developed as a programing library for Processing andcontained a set of pre-defined programing methods that allow the user to describe the lamp, and define the tool paths for all three materials. For the first version of the library, all design took place via textual programing. Within the Processing IDE one imports and initialize the controller class of the library, and uses it to call four main functions that determine the height, top width, middle width and bottom width of the lamp. These 4 parameters are used to determine the form of the lamp, by generating the equation of a parabola with 3 intersection points. By rotating this parabola round the y-axis, it was possible to generate a closed 3-dimensional ellipsoid form.  The library also provided access to an additional set of methods that control over a number of other parameters in describing the form of the lamp, including the number of sides, the resolution of the curve and the position of the internal structural supports.  When the code is compiled, the application displays a graphic view containing 3d wireframe model of the current lamp as defined by the parameters. 


		Workflow description
		Workshop
		Workshop results
\subsection{Discussion}

Because of their prior expertise, the experiences of the majority of the participants in the first study are not indicative of the feasibility of Codeable Objects for novice programmers. Their experiences provide valuable contrast to the experience of the novice coders in the successive workshops however, and provide important information about the usability and workflow of the software. Despite their experience in programing however, the experienced programmers in the lamp workshop exhibited limited knowledge of computational design prior to the start of the workshop. When asked in the pre-workshop surveys how they thought programing, design and craft could be combined, the general response was either uncertain, or as method to create dynamic interactivity, rather than a tool for the design of form and pattern:

\textit{``You can combine software and hardware and make craft more dynamic (e.g. sensors). [Lamp Participant pre 1]}

\textit{``[Programing] gives [you] the ability to make something dynamic. [Lamp Participant pre 3]}

Following the workshop, the participants were generally pleased with the creative affordances of the tool, and described how the software enabled them to expand their programing abilities to the realm of art and craft with greater success: 

\textit{``I think programming makes designing more accessible because you don't have to be able to draw or paint. 
[Lamp participant post 4] }

\textit{``I love the idea of being able to combine my interest in programming for creative expressions. 
[Lamp participant post 6]}

There was also an awareness among several participants about the practical benefits of combining computational design and digital fabrication:

\textit{``I understand now how programming can be used for quick prototyping and mockups that can be used to inform final design decisions. This is easy [and] helpful when using physical materials where mistakes can be costly. 
[Lamp participant post 2]}

\textit{``Using programming in the design process adds some exciting and unique capabilities over traditional design and crafting, including mixing in different algorithm and ideas from other existing software, and rapid prototyping of complex designs."[Lamp participant post 6]}

From these responses, it is apparent that even among experienced programmers, algorithmic craft has the potential to expand people's understanding of the applications of programming and motivate them to apply computation to other forms of production and expression. There were also elements of the process and tool that were problematic for the participants. It became immediately clear during the workshop that textual programing was not the optimal method of modifying the form of the lamp. Many of the participants became frustrated about having to set the parameters and then wait for the compilation process to complete before they could view the resulting form. This issue was addressed partly in subsequent versions of the tool by replacing the textual parameters with a set of sliders in the compiled application, which would adjust the form in real time, across each of the views (figure: \ref{fig:slider_interface}.)
\begin{center}
\begin{figure}[h!]
\includegraphics[width=6.5in]{images/slider_interface.png}
\caption{Revised graphic view with sliders}
\label{fig:slider_interface}
\end{figure}
\end{center}

The textual programing method proved to be useful in the context of the point specification for the pattern. The simple method of specifying points in a programing context, allowed for the wide degree of variation and approaches in the resulting designs. If the tool had relied on a more standard set of graphical user interface(UI) components, like sliders to control the point generation, it is doubtful that the same range and creativity could have been achieved. On the other hand, it was clear that the less experienced programmers had more difficulty deliberately designing the patterns of their lamps, and relied primarily on adjusting and remixing existing examples. 

Several participants also put forth detailed critiques of the programming process, which brought into focus concerns about the practice of computational design itself. One participant reacted against defining the generative qualities of the Voronoi diagram patterns as a design method:

 \textit{`"Changing the parameters didn't always generate the pattern you have in mind. It was more like generating a few semi-random patterns and you choose one that looks good. It is rather a trying-and-choosing rather than designing /making something you planned to have. I think "design" involves "intention" and "planning." Programming, crafting, and design should be combined in the way that entails prior planning and intentions as opposed to cutting together the semi-random choices, which could be good but I wouldn't call that design. [Lamp participant post 6]"}
 
This comment addresses the concern that the attributes of randomness and generativity do not automatically lead to optimal or good design decisions. Some deciding factor has to play a role in the process, but the designer�s role in the deciding process is often ambiguous. This criticism touches on a core debate about the role of conscious design and the restriction of intuitive creativity in computational practices overall, however it is particularly relevant to computational design. The emergence of comments like this are encouraging, because they reflect the engagement of the participants, not just with the task at hand, but in a critical evaluation of the  creative implications of this form of creation. This comment however also highlights a key restriction of Codeable Objects. While it is ambiguous to the extent at which adjusting the parameters and input values to a system constitutes design, the task of defining the algorithms which shape the system itself are decidedly a form of design. With Codeable Objects however, the user is unable to modify the core algorithms which define the range of forms and patterns that are possible, unless they alter the source code of the library itself. When evaluated as a tool for algorithmic craft, Codeable Objects could have done a better job of supporting some of the deeper components of computational design, in particular, the algorithmic abstraction of personal styles and aesthetics. The stylistic limitations contained in the tool most likely contributed to the high success rate in project completion, and the general attractiveness of the resulting projects, but the experience of the workshop, provided the motivation for future tools to have better balance of stylistic and computational openness and accessibility for new programmers. 

One last defining component of the Codeable Objects pilot workshop was the stark contrast between the nature of the challenges in the computational design and digital fabrication components and the crafting component. The difficulties people experienced while designing and fabricating their projects were often discrete, for example correcting for mathematical error in coordinate placement, or having the incorrect setting on the laser cutter. More complex problems sometimes arose in these contexts as well, such as confusing on the principles behind some of the more complex point generation algorithms, or the programming aspects in general, however they were seemingly aspects that could be addressed through verbal instruction and explanation. The challenges encountered in the crafting session were of a different quality, concerning the best techniques for assembling the parts so that the resulting product maintained an attractive appearance. Most participants were surprised at the amount of time required to complete the physical assembly, and were often frustrated when variations in the crafting process violated the precision and perfection of the digital design, and laser cut parts. Rather than addressing these difficulties through instruction and explanation, these were challenges that are best overcome through practice and experience with the materials. This contrast poses an interesting conflict for Algorithmic Craft in general. While both computational and hand crafting methods benefit from experience and practice, the approaches for solving problems differ significantly in programing and hand crafting. Programming often requires an analytical approach with an emphasis on consistency and regularity, whereas crafting requires a more intuitive process of responding and adjusting one's technique while in direct contact with the materials. How then, can algorithmic tools be presented in a manner that accustoms users to operating in both discreet and intuitive contexts, and how can these two modes of working inform one another?

	\section{Soft Objects}
After an evaluation of the successes and limitations of the Codeable Objects library I made an effort to expand the library in a way that would allow for a broader range of computational design approaches and end products. In particular, I was interested in exploring the domain of algorithmically crafted garments and fashion accessories. Fashion is an exiting domain to connect to computation, because it appeals to groups of people who are often under-represented in computer science, particularly women and girls. In addition, because garments and accessories are wearable, computational fashion design requires the programmer to consider questions of comfort, sizing and personal taste and and style; a set of concerns not often associated with most computer programs. To explore computational fashion design in the context of algorithm craft, I expanded the Codeable Objects tool into a more general programing library named SoftObjects and evaluated it over a 10 day workshop with young people.


\subsection{Motivation}


\subsection{Tool description}
The Soft Objects library contains a set of methods that allows users to draw shapes and patterns and then export those shapes and patterns in a vector-file format that is compatible with x-y axis digital-fabrication machines. Similar to CodeableObjects, to use the library, a user imports it into the Processing environment and then writes and compiles code using the Processing editor. Soft Objects allows users to define and manipulate basic geometric primitives such as Points, Lines, Curves and Polygons. These primitives can then be collected within Pattern and Shape objects�structures designed to capture surface decoration and 2D structure, respectively�to form increasingly complex designs. 

Soft Objects is formulated on an Object Oriented Programming (OOP) paradigm, which lets users create and manipulate collections of geometric primitives�Patterns and Shapes. This structure differs from Processing�s drawing API, which uses a functional programing approach. The structure of Soft Objects enables users to simultaneously apply transformations to all of the elements in a collection that make up a complex pattern or shape. It is also possible to import scalar vector graphics files (SVGs) to incorporate pre-drawn designs as elements within a pattern or as a container for existing patterns.

\begin{center}
\begin{figure}[h!]
\fbox{\includegraphics[width=6.5in]{images/primitive_syntax.png}}
\caption{Soft Objects primitives}
\label{fig:softobjects_primitives}
\end{figure}
\end{center}
		

Users are presented with a 2D preview of their designs when they compile their code. Soft Objects supports a variety of digital-fabrication machines by allowing users to save designs to vector portable document format (PDF) files. PDFs can be used by different production tools, including ink-jet printers, vinyl cutters, laser cutters, and computationally controlled embroidery machines. Output from Soft Objects can be fabricated on essentially any x-y axis tool. 3D structures can be created by assembling fabricated pieces. Figure \ref{fig:softobjects_workflow} demonstrates the workflow from code to a finished object. 
The Soft Objects library also contains a collection of pre-defined algorithmic patterns that can be initialized, including Voronoi diagrams, Koch curves, and L-Systems, and an extensive set of example programs that users can modify and combine to produce individual results.		

\begin{center}
\begin{figure}[h!]
\includegraphics[width=6.5in]{images/softobjects_workflow.jpg}
\caption{Soft Objects workflow}
\label{fig:softobjects_workflow}
\end{figure}
\end{center}
		
\subsection{Workshop}
The evaluation of Soft Objects was conducted during a 10-day workshop with a representative group of participants�eight young adults, aged 11-17, 75\% male and 25\% female. A significant majority (88\%) stated in pre-surveys that they had little or no prior experience in programming, and only one participant had prior experience in Processing. All of the participants indicated some level of prior experience in art, design, or craft. Most attributed their design or craft experience to art or drawing classes.

The workshop was conducted at the Nuvu Magnet Innovation Center for Young Minds. Participants were given 10 days to conceptualize and construct a garment using a combination of computational design, digital fabrication, and traditional sewing and crafting. The second study was more open than the first; participants could produce any type of garment they wished as long as components of it were computationally designed and digitally fabricated. During the workshop, participants were introduced to Soft Objects and the concept of computational fashion through a multi-step process that engaged participants in different levels of programming through the construction of different garments and accessories. First, participants were provided with a small set of example programs similar to the lamp workshop. This step allowed them to manipulate a core set of parameters to generate the pattern and form of a scarf, which they then cut on the laser cutter (Fig 3). 

Second, participants were instructed in a number of primary programing concepts, including iteration, function definition, and the use of variables and primitive data-types. During this instruction, participants were guided through the process of independently using Soft Objects and generating their own programs from scratch. They used these programs to create a design for a wooden bracelet (Fig. 3), which was then laser cut and assembled. After these two initiation activities, these participants were asked to conceive their own garments and provided with the resources to design, prototype, and craft finished garments.


\subsection{Results}
Participants in the fashion workshop were successful in using programing and digital fabrication to design and produce finished garments. During the initiation activities, participants independently wrote and compiled programs of their own and produced physical products based on the design generated from that program. Furthermore, with assistance from the instructors, the participants were able to apply more sophisticated programing methods to produce a diverse set of final products (Fig 4). One pair of students developed an �armor dress� by writing a program that geometrically described a single �scale� shape, imported a dress pattern from Illustrator and filled it with rows of scales that corresponded with the dimensions of the dress. Another pair created a geometrically inspired dress with a patterning of different-sized octagons and squares that were laser cut from starched fabric. Another student created American-flag-inspired pants using a program that generated random orderings of red and blue stripes on a white background. One group that was less interested in the process of sewing clothing created a program that generated a recursive virus-like pattern and then screen-printed the pattern on pre-made sweatshirts and t-shirts. 

On the post survey, when asked if they �were able to complete a finished project to their satisfaction,� 100\% of the fashion participants responded yes. The resultant garments were attractive and functional, indicated by the fact that participants from the fashion workshop kept and wore their creations. Direct comparison of the pre-and post-workshop surveys also demonstrated that on average, participants in the fashion workshop indicated their interest in crafting increased after the workshop, as did their enjoyment of the design process. Eighty-eight percent of the participants in the fashion workshop indicated that they felt more comfortable programing after the workshop than before.

		
	\section{DressCode}
		Motivation and design principles
		Tool Description
		Workflow description
		Workshop
		Workshop results
		Curriculum building
		Curriculum results
