%% This is an example first chapter.  You should put chapter/appendix that you
%% write into a separate file, and add a line \include{yourfilename} to
%% main.tex, where `yourfilename.tex' is the name of the chapter/appendix file.
%% You can process specific files by typing their names in at the 
%% \files=
%% prompt when you run the file main.tex through LaTeX.
\chapter{Introduction}

Computation is a driving force in our world. The power and ubiquity of  modern computer systems have made the skill of computer programming (hereafter programming) relevant to a wide range of human studies and disciplines. Most commonly, programming is viewed as an essential component of science, engineering, and business related applications\cite{resnick1}. As a result, many nascent programmers view programming as highly specialized, difficult, inaccessible, and only relevant as a career path in those particular fields. In reality, computation is a broad discipline with many applications, ranging from professional to personal. Foremost programing can serve as a medium for personal expression, through applications in art and design. With the emergence of digital fabrication technology, In addition, programing provides the means to design and produce useful objects and devices, not only on at an industrial scale, but also on a personal and individual scale \cite{mellis_thesis}. When programing is used create unique, functional physical objects, new possibilities emerge in the way people design,  the types of objects people create, and role programing can play in peoples' lives.  Computational design, the practice of programming to create form, structure and ornamentation, is a new way to design. When paired with digital fabrication technology, computational design allows people to make physical objects by writing code. My objective is to examine the combination of computational design, digital fabrication and traditional arts and crafts for the production of functional decorative objects. I define this domain with the term algorithmic craft. The process of bridging the spaces between textual programming language, visual design, and physical construction however, is not self-evident and raises many practical and theoretical questions.  What are the important design principles to consider when creating  programming environments for physical design? How do we compellingly link textual code with visual designs and what are the appropriate intersection points between textual manipulation and visual manipulation? What support is required to help people move back and forth from programming to building real objects in a way that is comfortable, expressive and pleasurable? How can we remove the technical challenges in translating code into an object that can be successfully fabricated, while still supporting a wide variety of design styles, aesthetics and approaches? Finally, how can we interlink the often disparate processes of physical prototyping with digital design and programming in a way that creatively reinforces both physical and virtual modes of working?

This thesis explores the affordances of algorithmic craft as a form of casual creative engagement in programing, craft and fabrication. As a part of my study of algorithmic craft, I developed three different software tools to support practitioners in this domain: Codeable Objects, Soft Objects and DressCode. Codeable objects is a programing library that helps people design and fabricate laser-cut lamps. Soft Objects is an extension of Codeable Objects that allows people to use programing to create forms and patterns for fashion and garment design. Finally DressCode is a combined graphic design and programing environment with a custom programing language developed to support open-ended computational design for digital fabrication. Each of these tools were evaluated in workshops where people used them in combination with 2-axis fabrication machines to produce physical artifacts. Through the workshops, I examined different approaches to combining programing and visual design for novice programmers. In addition, I evaluated how the materials and approaches of various crafting techniques could be combined with the form and structure of Computer Aided Design (CAD). The lessons learned in each workshop were used to inform the development of the following tool. This thesis presents the evolution of these three tools, and demonstrates the rationale their design through user testing and feedback. I conclude with a discussion of some of the unique affordances of Algorithmic Craft, and the possibilities it offers for how we design, construct and craft artifacts in years to come.

