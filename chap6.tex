%% This is an example first chapter. You should put chapter/appendix that you
%% write into a separate file, and add a line \include{yourfilename} to
%% main.tex, where `yourfilename.tex' is the name of the chapter/appendix file.
%% You can process specific files by typing their names in at the 
%% \files=
%% prompt when you run the file main.tex through LaTeX.
\chapter{Soft Objects}
	
	\begin{center}
\includegraphics[width=6.5in]{images/fashion_show.png}
\end{center}
	
After an evaluation of the successes and limitations of the Codeable Objects library, I made an effort to modify the library in a way that would allow for a broader range of computational-design approaches and end products. In particular, I was interested in exploring the domain of algorithmically crafted garments and fashion accessories. Fashion is an exciting domain to connect to computation because it appeals to groups of people who are often under-represented in computer science, particularly females. Computational fashion design resonates well with algorithmic craft because it offers the opportunity to apply digital fabrication to textiles and fabrics, and because it introduces sewing and pattern-making as components in the construction process. Because garments and fashion accessories are created to be worn, computational fashion design requires the creator to consider questions of comfort, sizing and personal taste and and style when writing code.To support computational fashion design in the context of algorithm craft, I expanded Codeable Objects into a general programming library entitled Soft Objects and evaluated it over a 10-day workshop with young designers.


\section{Motivation}
With the growth in public awareness of digital fabrication, enthusiasm for fashion applications of digital fabrication technology is profuse. Much of this excitement is directed towards 3D-printed wearables and textiles. In July 2010, Iris Van Herpen released her Crystallization collection, which featured her first computationally designed, 3D-printed piece, marking the first time a 3D-printed garment had appeared on a fashion runway \cite{herpen}. Van Herpen and many other fashion designers continued using 3D printing as a medium for fashion. As a result, computationally designed, digitally fabricated fashion is often synonymous with 3D printing. The 3D-printed garments and accessories produced by professional designers like Van Herpen serve as inspiration for the future of digital fabrication and demonstrate wearable forms that would be impossible to create through any other means. For the average person, however, computationally designed and digitally fabricated garments of this nature present considerable limitations. Given current technology, cost and material, the majority of 3D-printed garments are impractical for every-day wear and require advanced fabrication techniques that are unavailable to many non-professional designers. Garments like the N.12 bikini designed by Continuum \cite{n12} are intended to be ready-to-wear, and available to consumers, but for the time being they are limited in scale, and still are costly. The construction of 3D-printed garments of this form appear to have little in common with existing methods of garment production, like sewing, knitting and embroidery. Admittedly, garments of this nature are inspirational and groundbreaking, yet individuals with sewing, knitting, weaving or others skills with textile manipulation may perceive their interests to be incompatible such forms of computational design and digital fabrication.

\begin{center}
\begin{figure}[h!]
\includegraphics[width=6.5in]{images/3D_prinited_high_fashion.png}
\caption{ 3D-printed fashion (from left to right: Crystallization 3D Top by Iris Van Herpen, Drape Dress by Janne Kyttanen, N12 Bikini by Continuum Fashion, Strvct shoe by Continuum Fashion}
\label{fig:high_fashion}
\end{figure}
\end{center}

Although perhaps less publicized than 3D-printed fashion, other  designers are merging fashion with computation and digital fabrication in a way that blends new technology with established approaches. Diana Eng's Laser Lace tee collection contains laser-cut machine-washable t-shirts with floral-inspired iconography. Further, her Fibonacci scarf is created through traditional knitting techniques, meshed with a Fibonacci knit pattern. Eunsuk Hur's modular fashion pieces are inspired by tessellations and fractal geometry. By creating garments through laser-cut interlocking pieces, Hur's stated aim is to produce items that are robust and durable, and also provide an opportunity for the user to engage in the creative process (figure:\ref{accessible_fashion}.)

These examples demonstrate a use of computational fashion design that is compatible with non-digital interests, and skill sets. Subtractive fabrication machines, like laser and vinyl cutters are dominant in this type of work because they work with traditional materials and can produce garments at  lower costs and larger scales than 3D printers. Variation in materials also can translate to a wide set of possibilities for hand crafting as well and different aesthetics and styles. It should be noted that accessible forms of 3D printing can produce compelling wearable objects; however the products produced are generally at the scale of jewelry and small accessories. Garments designed and produced as a blend of digital fabrication and textile materials and construction processes correspond well with the values and practices of algorithmic craft. The Soft Objects programming library was designed to allow new programmers to design the forms and aesthetic components of fashion accessories and garments through programming, followed by the physical construction of the garments using subtractive forms of digital fabrication and sewing. 

\begin{center}
\begin{figure}[h!]
\includegraphics[width=6.5in]{images/accessible_fashion.png}
\caption{"Ready-to-wear" computational fashion (from left to right: Fibonacci Scarf by Diana Eng, Biomimicry laser-cut bracelet by Stefanie Nieuwenhuyse, Laser Lace All-Over Tee by Diana Eng, Interstice bracelet by Nervous Systems, Modular Fashion by Eunsuk Hur}
\label{fig:accessible_fashion}
\end{figure}
\end{center}
		

\section{Tool Description}
The Soft Objects library contains a set of methods that allows users to draw shapes and patterns and then export those shapes and patterns in a vector format that is compatible with x-y axis digital-fabrication machines. Similar to Codeable Objects, Soft Objects is used within the Processing programming environment and contains a set of programing methods that enable the design of visual forms and patterns. SoftObjects allows users to define and manipulate basic geometric primitives such as points, lines, curves, and polygons. These primitives can then be collected within Pattern and Shape objects---structures designed to capture surface decoration and 2D structure, respectively---to form increasingly complex designs. Soft Objects is formulated on an Object Oriented Programming (OOP) paradigm, which lets users create and manipulate collections of geometric primitives---Patterns and Shapes. This structure differs from Processing's drawing API, which uses a functional programming approach. Users are presented with a 2D preview of their designs when they compile their code. The structure of Soft Objects enables users to simultaneously apply transformations to all of the elements in a collection that make up a complex pattern or shape. The objective behind this structure is to open the design possibilities in a format that was suitable for creating complex 2D designs and aesthetic patterns for clothing and accessories, while ensuring that a user's designs would be compatible with subtractive fabrication. To simplify the process of garment creation, Soft Objects included functionality to import existing cut patterns as scalar vector graphics files (SVGs). This allows users to merge programmatically generated designs with patterns for pre-sized shirts, dresses and pants.

\begin{center}
\begin{figure}[h!]
\fbox{\includegraphics[width=6.5in]{images/primitive_syntax.png}}
\caption{Soft Objects primitives.}
\label{fig:softobjects_primitives}
\end{figure}
\end{center}
		
Soft Objects supports a variety of digital-fabrication machines by allowing users to save designs to PDF files. Output from Soft Objects can be fabricated on essentially any x-y axis tool. 3D structures can be created by assembling fabricated pieces. Figure \ref{fig:softobjects_workflow} demonstrates the workflow from code to a finished object. 
The Soft Objects library also contains a collection of pre-defined algorithmic patterns that can be initialized, including Voronoi diagrams, Koch curves, and L-Systems. It also includes an extensive set of example programs that users can modify and combine to produce individual results. These examples and pre-defined algorithms are useful to  expose new programmers to some of the complexity and variability that is possible through computational design. They also  demonstrate the abstract qualities of computation, without immediately requiring people to learn substantial amounts of programing structure and syntax. 		

\begin{center}
\begin{figure}[h!]
\includegraphics[width=6.5in]{images/softobjects_workflow.jpg}
\caption{Soft Objects workflow.}
\label{fig:softobjects_workflow}
\end{figure}
\end{center}
		
\section{Workshop}
The evaluation of Soft Objects was conducted during a 10-day workshop with a representative group of participants---eight young adults, ranging in age from 11 to 17 and composed of 75\% male and 25\% female. A significant majority (88\%) stated in pre-surveys that they had little or no prior experience in programming, and only one participant had prior experience in Processing. All of the participants indicated some level of prior experience in art, design, or craft. Most attributed their design or craft experience to art or drawing classes. The workshop was conducted at the Nuvu Magnet Innovation Center for Young Minds. Participants were given 10 days to conceptualize and construct a garment using a combination of computational design, digital fabrication, sewing and crafting. The workshop was more open-ended than the Codeable Objects workshop because participants could produce any type of garment they wished as long as components were computationally designed and digitally fabricated. During the workshop, participants were introduced to Soft Objects and the concept of computational fashion in a multi-step process that engaged participants in different levels of programming through the construction of different garments and accessories. First, participants were provided with a small set of example programs similar to the lamp workshop. This step allowed them to manipulate a core set of parameters to generate the pattern and form of a scarf, which they then cut on a laser cutter (figure: \ref{fig:scarves_bracelets}.) 

\begin{center}
\begin{figure}[h!]
\includegraphics[width=6.5in]{images/scarves_bracelets.png}
\caption{Bracelets and scarves from preliminary activities.}
\label{fig:scarves_bracelets}
\end{figure}
\end{center}


Second, participants were instructed in a number of primary programming concepts, including iteration, function definition, and the use of variables and primitive data-types. During this instruction, participants were guided through the process of independently using Soft Objects and generating their own programs from scratch. They used these programs to create a design for a wooden bracelet (figure: \ref{fig:scarves_bracelets}), which was then laser cut and assembled. These initiation activities demonstrated the stylistic affordances and design properties of computational design and digital fabrication and  providing participants with the practical foundation to begin conceiving of and executing their own ideas. During the remainder of the workshop, the participants were asked to conceive of concepts and designs for their own garments. They were provided with the resources to design, prototype, and craft finished artifacts.

\section{Results}
Participants in the fashion workshop were successful in using programming and digital fabrication to design and produce finished garments. During initiation activities, participants independently wrote and compiled programs of  and produced physical products based on the design generated from their programs. With assistance the participants applied sophisticated programming methods to produce a diverse set of final products (figure: \ref{fig:fashion_results}). One pair of students developed an ``armor dress" by writing a program that geometrically described a single ``scale" shape, imported a dress pattern from Illustrator and filled it with rows of scales that corresponded with the dimensions of the dress. Another pair created a geometrically inspired dress with octagons and squares of different sizes that were laser cut from starched fabric. Another student created American-flag-inspired pants using a program that generated random orderings of red and blue stripes on a white background. One group that was less interested in the process of sewing clothing than other students created a program that generated a recursive virus-like pattern. Then they used screen printing to reproduce the pattern on pre-made  sweatshirts and t-shirts.

\begin{center}
\begin{figure}[h!]
\includegraphics[width=6.5in]{images/fashion_results.png}
\caption{Completed garments (from left to right: octagon dress, flag pants, samurai dress, viral sweatshirt)}
\label{fig:fashion_results}
\end{figure}
\end{center}


On the post survey, when asked if they ``were able to complete a finished project to their satisfaction," 100\% of the participants responded in the affirmative. Their garments were attractive and functional, indicated by the fact that participants from the fashion workshop kept and wore their creations. Direct comparison of the pre-and post-workshop surveys also demonstrated that on average participants interest in crafting increased as a result of the workshop, as did their enjoyment of the design process. Eighty-eight\% of the participants in the workshop indicated that they felt more comfortable programming after the workshop than before.

\section{Discussion}
The Soft Objects workshop occurred over a longer time period than the Codeable Objects workshop, and explored a wide range of crafting and computational techniques. As a result, I had an opportunity to spend considerable amounts of time observing and talking with the participants. Because the participants were novice programers, their experiences better reflect the target demographic of this research than in the Codeable Objects study. Through this workshop I confirmed that algorithmic craft activities can actively support the expression of personal identity among young designers. The workshop activities fostered feelings of confidence in programming and supported aesthetic and technological literacy. They also promoted a deep understanding of computation as evidenced by critiques of the participants and demonstrated the importance of physical prototypes in the design process. Finally, the workshop promoted a sustained engagement in programming.

 
\subsection{Identification as a Programmer}
Similar to the participants in the Codeable Objects workshop, the fashion participants began with vague ideas about the applications of computation. When asked in the pre-workshop surveys how they thought programming, art and craft could be combined, participants responded in writing in the following ways:
\begin{quotation}
\textit{``To be honest, I am not too sure how it would all be combined because I don't know much about programming."} 
\\Fashion Participant Y

\textit{``With programming, we can make programs do things for us."}
\\Fashion Participant J
\end{quotation}

In addition, the participants had almost no prior knowledge of computational design. Those participants who had some with prior programming experience associated it largely with interactivity and actuation:
\begin{quotation}
\textit{``I would love to combine things like T-shirts and speakers or other different types of technologies."}
\\Fashion Participant J
\end{quotation}

Following the completion of their final projects participants were much more descriptive about the applications of programming:

\begin{quotation}
\textit{``I think [programming and fashion] are really interesting but I never thought they could ever be together in one concept, and it's awesome that I know that now- that you can design aesthetically pleasing things from coding."}
\\Fashion participant K


\textit{``I've never thought of programming as physical; I thought it was only in computers. But then when we made the scarves and stuff, I thought that was really fun."}
\\Fashion participant M
\end{quotation}

Many people also expressed a growing confidence in their ability as programmers. During the interviews, several separately stated that that while they did not feel completely comfortable programming independently, the experience made programming feel significantly more accessible:
\begin{quotation}
\textit{``For someone who never had any programming explained to them before, when you look at [computational drawing examples] it feels really inaccessible, but now that I've been taught a little bit of [programming], I can kind of crack at the walls a little bit and understand how it works, and that makes it more accessible to me."}
\\Fashion participant S

\textit{``Even though now I'm not really a Processing expert now, I've just experienced it and it's not as scary to me, like the idea of coding, you just kind of have to learn some stuff and practice it more, but I think I definitely understand the concept of it."}
\\Fashion participant K
\end{quotation}
Finally, the programmers in the fashion workshop expressed feelings of pride and a sense of accomplishment in their new-found programming skills:
\begin{quotation}
\textit{``I'm actually kind of proud. I know what's going on. It feels different... I just thought [programming] was just about these huge programs that you have to piece together, and that you have to be really, really smart to do it, but I can do it."}
\\Fashion participant P

\textit{``I think it was really cool that we used [programming] for fashion, cause I think a lot of people might think people who do fashion aren't really smart or something, and then they think that people who design code are like brilliant coders and can do really awesome stuff with it."}
\\Fashion participant K
\end{quotation}

 The confidence, sense of belonging, and personal agency demonstrated in these comments stand in contrast to popular views of programming as specialized and inaccessible. These sentiments indicate that introducing programming in the context of algorithmic crafting not only has the potential to change people's understanding of the relevance and applications of computation, but also promote a personal awareness of technological literacy and competence. 
 
\subsection{Application of Computational Affordances}
Aside from a general understanding of the potential applications of computation, participants were able to leverage the advantages of computational design in their final projects. For example, the octagon dress used parametric design principles, which was evident because participants were able to change the size and orientation of the octagons in the dress by modifying several parameters at the start of their program to affect the entire pattern, rather than rotating and adjusting each shape independently. The samurai dress took advantage of the computational properties of precision and automation. With help from an instructor, the creators programmatically generated an individual vector file for each row of scales of the dress, expediting the production process (figure:\ref{fig:samurai_dress_progression}.) The samurai dress is particularly interesting because it shows a remarkable  transition from the design expressed in original concept art by the participant and the final garment, demonstrating  adherence to the person's creative vision through computation. Lastly, the viral shirt, demonstrated an application of generativity. The aesthetic of the viral pattern was produced using a weighted random-number generator to determine the number and length of the branches in each recursion of the pattern. Although the implementation of the weighted number generator was facilitated with help from the instructor, the participants came up with the idea of using it on their own.
\begin{center}
\begin{figure}[h!]
\includegraphics[width=6.5in]{images/samurai_dress_progression.png}
\caption{Progression of samurai dress (from left to right: concept sketch, computationally generated pattern,laser-cut components of final garment)}
\label{fig:samurai_dress_progression}
\end{figure}
\end{center}

Many of the participants demonstrated an understanding of the rationale behind the methods they were using. One of the creators of the armor dress compared the process of programming the design of the dress to that of manually drawing it: 

\textit{``With drawing you can achieve everything programming can, but I would prefer to program it. [programming] can be pretty convenient...the computer is helping me. Like if you want to make pizza, the computer is like a pre-made crust."}[Fashion participant E]

Participants also articulated an understanding of specific programming functionality. One participant described the point at which she understood the application 
of parameterization:

\textit{``One moment that stuck out was when you helped me make a code with original geometry that could be changed so that when you changed one thing it changed everything and that was cool because I felt like I actually made something that could be changed and then applied."}
[Fashion participant K]

The ability to understand and describe how computation supports creative objectives is essential in motivating an individual to spend time learning and implementing these methods. Once the aesthetic possibilities of the recursive viral pattern were apparent to the designers, they became engaged in understanding the underlying algorithm so that they could produce a pattern to their exact specifications. Algorithmic craft can provide the motivation for participants to tackle complex computational problems with sophisticated approaches when problems are clearly grounded within the design objectives of the individual. A continual challenge in engaging new programmers in algorithmic craft is  selecting programming approaches that allow for compelling aesthetic and design possibilities but remain approachable for first-time coders. 

\subsection{Aesthetics and Identity Expressed Through Code}
The fashion workshop provided the opportunity for participants to use computational aesthetics as a way to express their personal style. Fashion can serve as a means of self-expression and for conveying one's identity. Discussions about fashion conducted at the start of the workshop indicated that participants were aware of the connection between fashion and identity and were eager for opportunities to create clothes that expressed their style. As a result, the majority of the garments created in the workshop contained an expression of the fashion sense of the participants who created them. For example, the participant who created the flag pants was very explicit that the pants have some form of an American flag motif, but not resemble the traditional, and as he put it ``tacky" flag pants that he commonly saw. He wanted his flag pants to be ``something that he would actually want to wear." His programming choices were made in direct consideration of his desire to create a pair of pants that he felt were fashionable.

When asked about the experience of making and designing his pants he said:

\textit{``[The workshop] definitely changed my impression of making clothes, I thought it was pretty quick to make clothes, but it actually takes a long time, and it's also really fun. I love the fabric I made."} [Fashion participant M]

His enthusiasm also was evident in the fact that after the pants were complete, he tried them on and wore them for the remainder of the workshop. This level of enthusiasm was common among participants; they all proudly modeled their creations, and many of them wore them home. This behavior suggests a relationship between the decisions made in a programming context and the participant's desires to express their visual identity. The participants were selective in the code they wrote to design their garments because they intended to wear the garments, and as a result, be represented by them. This powerful affective relationship between computation, design, and self-expression provides a natural way to engage people in programming and design by supporting their personal interests. 

\subsection{Physical and Digital Connections}

One of the challenges of algorithmic craft, alluded to in the Codeable Objects discussion, is that the practitioner must work between digital designs with physical materials and processes. Physical prototypes often serve as a key point of transition between these spaces. In the fashion workshop, prototyping played an important role, and demonstrated how computational tools can support and sometimes hinder the prototyping process. The focus on fashion made it possible to supply the participants with large amounts of inexpensive test fabric. The laser cutter could cut fabric much more quickly than thick materials, which allowed participants to produce numerous prototypes of their projects before creating a final piece. Most groups produced two or three prototypes, with one participant creating six iterations of a single jacket. This rapid production process formed a direct connection between discoveries made in the physical prototyping space and decisions in the programming realm. 
In the case of the octagon dress, (figure:\ref{fig:fashion_results}), the participants first cut test rows of octagons to determine the appropriate scale, then adjusted their design by modifying their program.Once they cut out a second more -omplete version of the dress, they rotated one of the shoulder straps on the physical prototype and formed an idea for a one-sided shoulder strap. They implemented this design change in the digital version of the dress by making additional changes to the size and rotation of the shapes defined in the code. When asked about this process in the interview one of the participants said:

\textit{``I think it was really fun that we got to do a prototype first because then if you don't like it, you don't feel a lot of pressure because you can make it again really fast, and there's no stress because if it doesn't turn out well, then it's not your final project.} [Fashion Participant K]

The combination of programming, rapid fabrication, and physical construction allowed for a design approach that transitioned from programming to fabrication to programming adjustments based on the fabricated elements, and then back to fabrication. This iterative approach resulted in a closely linked cycle of physical and digital engagement. 

 \subsection{Enthusiasm in Crafting and Coding}
One of the most encouraging aspects of following the fashion workshop, was the participant's enthusiasm and desire to continue making. Participants talked extensively about what they would like to make in future with programming, stating that they would like to continue making clothing, or other personal functional items like furniture and ``things they could use around the house." The experience of both sets of workshop participants also demonstrated the ability of these techniques to produce objects that were designed to complement personal items and living spaces. When asked what she would like to make if she continued to program, one participant responded: 

\textit{``Things like we're making now, things that you would want to keep or use, things that look nice as opposed to like computer games, or ``input-output" devices. I think those are fun, but it's not as cool as things that you can hold in your hand. I actually hung up the scarf I made in my room, and now I can be like ``I made this on Processing" and people will be like what? It's cool! "}[Fashion participant K]

This enthusiasm, combined with the high potential for individual expression and sense of accomplishment encouraged meto continue exploring fashion and garment production as topic space for algorithmic craft. While the fashion workshop highlighted many positives in this sense, I also encountered several areas for improvement. 

\section{Limitations}
The most evident barriers in the Fashion workshop involved the syntactic challenges of programming. Many participants expressed a frustration with the syntax in both surveys and in-person interviews. Although the workshop participants were able to generate their own programs, they required more assistance from an instructor to write some of the commands. In addition, a feeling of needing to memorize programming syntax frequently translated to a sense of frustration. One participant stated in an interview:

\textit{``I couldn't memorize things, so it also was frustrating for me to always have to get you to help me write the code."} [Fashion participant K]

Many people requested some form of written "cheat sheet" that listed the key methods and how to use them. They also pointed out that they often had to write a lot of code (including import commands as well as setup and draw functions), even for simple tasks. Writing code for the first time is always challenging; however the high levels of frustration registered by the participants often focused on aspects of programming that were extraneous to the design itself. Because I wrote Soft Objects as a Processing library, it required that the syntax correspond to Java, a difficult language for beginners. Java is a general application language and has many syntactic requirements that are unnecessary for computational design applications. Based on this difficulty I concluded that future algorithmic crafting tools should explore domain-specific languages that directly applied to design and fabrication.
 
Along with difficulties with the programming syntax, participants struggled with some of the post-processing techniques. In order to be suitable for digital fabrication, many of the participant's computationally generated designs required some post processing in Illustrator. Usually this required using Illustrator's shape boolean functionality to merge shapes or expand outlines so that the vector paths would correspond to the desired cut pattern on the laser cutter (figure:\ref{fig:illustrator_example}.) Although the methods to perform these operations were simple, they needed to be repeated every time the design was modified programmatically. This was not only inefficient but sometimes prevented people from determining if their designs were feasible for fabrication when they were in the programming environment. This difficulty encouraged me to focus on ways of removing the need for Illustrator from the process altogether so that all design modifications could be initialized and updated programmatically.

Many participants  also struggled with the concept of prototyping. These struggles were evident in practical aspects, such as participants not saving their programs and digital design files (despite repeated reminders from the instructors to do so). Participants also frequently spent too much time on assembling their early prototypes, and were they frustrated when they realized they would have to make design changes and repeat some of the manual labor. Although I took pains in the workshop to introduce participants to the concept of prototyping, these instructions were not always absorbed. This may present an opportunity for future algorithmic crafting tools to contain specific features to encourage and support the physical prototyping process. One possibility is to include simulation tools that preview the constraints and behaviors of physical materials. The value of working with physical prototypes should also be supported, however, by features that allow designers to scale their files and fabricate first in miniature, and by software that allows easy access and management of multiple versions of a design throughout the design process.
	
Finally, as in the Codeable Objects workshop, participants were frustrated by the delay between adjusting their code and seeing the results. Although, not an absolute solution to the challenges in learning programming syntax, a programming environment with quick feedback could assist with the issue of syntactic challenges by providing novice users with improved ways to visualize the effect their syntactic changes have on their design. The implementation of background compiling, the process by which code is automatically compiled and executed as changes are made, has been applied successfully in several tools for novice programmers, including Scratch and Alice \cite{alice}, and more recently with Khan Academy \cite{khan}. My reliance on the Processing for Soft Objects made the incorporation of real-time compilation infeasible, however that change quickly became a goal for future tools.

\section{Summary}
Participants in the Soft Objects workshop were able to create personal wearable items that were beautiful, functional, and personally meaningful through computational design, digital fabrication and craft. During the workshop, the participants learned new skills in programing and digital fabrication, as well as techniques for pattern-making and sewing. The participants emerged with an awareness of the applications of computational design for digital fabrication, specific understanding of some of the primary elements of programming, and a desire to continue using computation to build their own objects. Although there are many areas for improvement in future tools, the Soft Object workshop primarily demonstrated the importance of incorporating approaches that further reduce the practical challenges of programming. Possible approaches include removing extraneous syntax, reducing the number of programing methods, providing rapid and informative visual feedback, and adding features that help users make their designs feasible for fabrication. The objective of these features is not to trivialize the process of algorithmic craft, but to remove barriers to independent and deliberate creation through computational design, and to ensure the successful translation of digital designs to physical forms that are compatible with craft practices. 