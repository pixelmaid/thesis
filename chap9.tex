%% This is an example first chapter.  You should put chapter/appendix that you
%% write into a separate file, and add a line \include{yourfilename} to
%% main.tex, where `yourfilename.tex' is the name of the chapter/appendix file.
%% You can process specific files by typing their names in at the 
%% \files=
%% prompt when you run the file main.tex through LaTeX.
\chapter{Conclusion and Future Directions}
	\section{Functional Properties of Algorithmic Crafting Tools}
	Based on my analysis of existing CAD and computational design tools, and my study of professional computational design and digital fabrication practices, I hypothesized that the following functional properties are necessary for an effective algorithmic craft software:
	\begin{itemize}
\item \textbf{Emphasis on computational design:} Programing should be the chief method of generating and manipulating designs, and this focus should be reflected in the interface of the software. 
\item \textbf{ Novice-oriented programing syntax:} The programing syntax and application programing interface (API) should be designed to accommodate individuals with no prior programing experience, and should be limited to methods and structures relevant to the task of design. 
\item \textbf{Support of sophisticated design techniques:} Although the programatic aspects of the program should seek to cater to new programmers, the tool should work for a range of design expertise and objectives:  from amateurs with an expressed interest in design and aesthetics, to those with prior design experience. 
\item \textbf {Design methods that facilitate digital fabrication:} The software should include programing and drawing methods that allow for the production of designs that are suitable for digital fabrication and support for exporting to relevant file formats. 
\item \textbf{ Prioritization of visualization and simulation:} Users  should be given ample visual feedback to inform their programming decisions.
\item \textbf{ Simple workflow from software to fabrication:} The transition from the design tool to the fabrication device should require as few intermediary steps as possible.
\item \textbf{ Free and open-source:} The software should be freely available, and able to function on multiple platforms with low requirements for computational processing power to afford high levels of access to casual users. If possible, the software should also be open source, in order to encourage the proliferation of additional novice oriented tools that can be developed for the specific needs of distinct user groups.
\item \textbf{Domain specificity:} In order to support first-time users, the tool should be constrained to a the design of a particular set of end products. Or, if the software is general purpose, it should be packaged with a set of well documented example designs and projects that clearly demonstrate its key applications. 
\item \textbf{Fabrication and craft-specific documentation:} In addition to the documentation of the application and programing language, the fabrication and crafting techniques intended for use with the software should be well-documented. This documentation should include details on suitable materials, fabrication machine access and settings, and tutorials on the craft components of example projects.
\end{itemize}
	
	Version Control (The loss of design)
	Better selection mechanisims
	Longer-term studies
Targeted audience (revised)
Importance of in-person instructors- in relation to craft related activities
