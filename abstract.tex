% $Log: abstract.tex,v $
% Revision 1.1  93/05/14  14:56:25  starflt
% Initial revision
% 
% Revision 1.1  90/05/04  10:41:01  lwvanels
% Initial revision
% 
%
%% The text of your abstract and nothing else (other than comments) goes here.
%% It will be single-spaced and the rest of the text that is supposed to go on
%% the abstract page will be generated by the abstractpage environment.  This
%% file should be \input (not \include 'd) from cover.tex.
The accessibility, diversity, and functionality of modern computer systems make computer programming (hereafter programming) useful in many realms of human study and advancement. Visual and physical art, craft, and design are interrelated domains that offer exciting possibilities when combined with programming. Unfortunately, use of programming is currently limited as a medium for art and design, especially by young adults and amateurs. Many potential users view programming as highly specialized, difficult, inaccessible, and only relevant as a career path in science, engineering or business fields, rather than as a mode of personal expression. Despite this perception, programming has the potential to correspond well with traditional, physical art-making practices. By forging a strong connection between programming and the design and fabrication of personally relevant physical objects, it may be possible to foster meaningful experiences in both programming and design for novice practitioners. The combination of digital fabrication technologies with computational design serves as one such connection.