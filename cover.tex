% -*-latex-*-
% 
% For questions, comments, concerns or complaints:
% thesis@mit.edu
% 
%
% $Log: cover.tex,v $
% Revision 1.8  2008/05/13 15:02:15  jdreed
% Degree month is June, not May.  Added note about prevdegrees.
% Arthur Smith's title updated
%
% Revision 1.7  2001/02/08 18:53:16  boojum
% changed some \newpages to \cleardoublepages
%
% Revision 1.6  1999/10/21 14:49:31  boojum
% changed comment referring to documentstyle
%
% Revision 1.5  1999/10/21 14:39:04  boojum
% *** empty log message ***
%
% Revision 1.4  1997/04/18  17:54:10  othomas
% added page numbers on abstract and cover, and made 1 abstract
% page the default rather than 2.  (anne hunter tells me this
% is the new institute standard.)
%
% Revision 1.4  1997/04/18  17:54:10  othomas
% added page numbers on abstract and cover, and made 1 abstract
% page the default rather than 2.  (anne hunter tells me this
% is the new institute standard.)
%
% Revision 1.3  93/05/17  17:06:29  starflt
% Added acknowledgements section (suggested by tompalka)
% 
% Revision 1.2  92/04/22  13:13:13  epeisach
% Fixes for 1991 course 6 requirements
% Phrase "and to grant others the right to do so" has been added to 
% permission clause
% Second copy of abstract is not counted as separate pages so numbering works
% out
% 
% Revision 1.1  92/04/22  13:08:20  epeisach

% NOTE:
% These templates make an effort to conform to the MIT Thesis specifications,
% however the specifications can change.  We recommend that you verify the
% layout of your title page with your thesis advisor and/or the MIT 
% Libraries before printing your final copy.
\title{Algorithmic Craft: Tools and Practices For Creating Useful and Decorative Objects With Code}

\author{Jennifer Jacobs}
% If you wish to list your previous degrees on the cover page, use the 
% previous degrees command:
%       \prevdegrees{A.A., Harvard University (1985)}
% You can use the \\ command to list multiple previous degrees
%       \prevdegrees{B.S., University of California (1978) \\
%                    S.M., Massachusetts Institute of Technology (1981)}
\department{Department of Media Arts and Sciences}

% If the thesis is for two degrees simultaneously, list them both
% separated by \and like this:
% \degree{Doctor of Philosophy \and Master of Science}
\degree{Masters of Science}

% As of the 2007-08 academic year, valid degree months are September, 
% February, or June.  The default is June.
\degreemonth{June}
\degreeyear{2013}
\thesisdate{June something, 2013}

%% By default, the thesis will be copyrighted to MIT.  If you need to copyright
%% the thesis to yourself, just specify the `vi' documentclass option.  If for
%% some reason you want to exactly specify the copyright notice text, you can
%% use the \copyrightnoticetext command.  
%\copyrightnoticetext{\copyright IBM, 1990.  Do not open till Xmas.}

% If there is more than one supervisor, use the \supervisor command
% once for each.
\supervisor{Leah Buechley}{Associate Professor of Media Arts and Sciences}
\supervisor{Mitchel Resnick}{ Academic Head, Program in Media Arts and Sciences}
\supervisor{Neri Oxman}{Assistant Professor of Media Arts and Sciences}

% This is the department committee chairman, not the thesis committee
% chairman.  You should replace this with your Department's Committee
% Chairman.
\chairman{...}{Chairman, Department Committee on Graduate Theses}

% Make the titlepage based on the above information.  If you need
% something special and can't use the standard form, you can specify
% the exact text of the titlepage yourself.  Put it in a titlepage
% environment and leave blank lines where you want vertical space.
% The spaces will be adjusted to fill the entire page.  The dotted
% lines for the signatures are made with the \signature command.
\maketitle

% The abstractpage environment sets up everything on the page except
% the text itself.  The title and other header material are put at the
% top of the page, and the supervisors are listed at the bottom.  A
% new page is begun both before and after.  Of course, an abstract may
% be more than one page itself.  If you need more control over the
% format of the page, you can use the abstract environment, which puts
% the word "Abstract" at the beginning and single spaces its text.

%% You can either \input (*not* \include) your abstract file, or you can put
%% the text of the abstract directly between the \begin{abstractpage} and
%% \end{abstractpage} commands.

% First copy: start a new page, and save the page number.
\cleardoublepage
% Uncomment the next line if you do NOT want a page number on your
% abstract and acknowledgments pages.
% \pagestyle{empty}
\setcounter{savepage}{\thepage}
\begin{abstractpage}
% $Log: abstract.tex,v $
% Revision 1.1  93/05/14  14:56:25  starflt
% Initial revision
% 
% Revision 1.1  90/05/04  10:41:01  lwvanels
% Initial revision
% 
%
%% The text of your abstract and nothing else (other than comments) goes here.
%% It will be single-spaced and the rest of the text that is supposed to go on
%% the abstract page will be generated by the abstractpage environment.  This
%% file should be \input (not \include 'd) from cover.tex.
The accessibility, diversity, and functionality of modern computer systems make computer programming useful in many realms of human study and advancement. Visual and physical art, craft, and design are interrelated domains that offer exciting possibilities when combined with programming. Unfortunately, the use of programming is currently limited as a medium for art and design for young people and amateur programmers. Many potential users view programming as highly specialized, difficult, inaccessible, and only relevant as a career path in science, engineering, or business fields, rather than as a mode of personal expression. Despite this perception, programming has the potential to correspond well with physical craft practices. By forging a strong connections between programming and the construction of personally relevant physical objects, it may be possible to foster meaningful creative experiences for non-professionals. The term algorithmic craft describes the combination of computational design, digital fabrication and craft to create functional artifacts. This thesis describes my work in developing a set of software tools that attempt to make the practice of algorithmic craft accessible for novice programers. Through it, I describe the design of each tool and discuss my experiences in engaging people in the creation of objects that are designed with programming, formed by machines, and shaped by hand. 
\end{abstractpage}


%% ----------------------------------------------------------------
% The "Funny Quote Page"
\pagestyle{empty}  % No headers or footers for the following pages

\null\vfill
% Now comes the "Funny Quote", written in italics
\textit{``The type of work which modern technology is most successful in reducing or even eliminating is skillful, productive work of human hands, in touch with real materials of one kind or another. In an advanced industrial society, such work has become exceedingly rare. A great part of the modern neurosis may be due to this very fact; for the human being enjoys nothing more to be creatively, usefully, productively engaged with both his hands and his brains."
}

\begin{flushright}
E.F. Schumacher, Small is Beautiful (1973)
\end{flushright}

\vfill\vfill\vfill\vfill\vfill\vfill\null
\clearpage  % Funny Quote page ended, start a new page
%% ----------------------------------------------------------------


% Additional copy: start a new page, and reset the page number.  This way,
% the second copy of the abstract is not counted as separate pages.
% Uncomment the next 6 lines if you need two copies of the abstract
% page.
% \setcounter{page}{\thesavepage}
% \begin{abstractpage}
% % $Log: abstract.tex,v $
% Revision 1.1  93/05/14  14:56:25  starflt
% Initial revision
% 
% Revision 1.1  90/05/04  10:41:01  lwvanels
% Initial revision
% 
%
%% The text of your abstract and nothing else (other than comments) goes here.
%% It will be single-spaced and the rest of the text that is supposed to go on
%% the abstract page will be generated by the abstractpage environment.  This
%% file should be \input (not \include 'd) from cover.tex.
The accessibility, diversity, and functionality of modern computer systems make computer programming useful in many realms of human study and advancement. Visual and physical art, craft, and design are interrelated domains that offer exciting possibilities when combined with programming. Unfortunately, the use of programming is currently limited as a medium for art and design for young people and amateur programmers. Many potential users view programming as highly specialized, difficult, inaccessible, and only relevant as a career path in science, engineering, or business fields, rather than as a mode of personal expression. Despite this perception, programming has the potential to correspond well with physical craft practices. By forging a strong connections between programming and the construction of personally relevant physical objects, it may be possible to foster meaningful creative experiences for non-professionals. The term algorithmic craft describes the combination of computational design, digital fabrication and craft to create functional artifacts. This thesis describes my work in developing a set of software tools that attempt to make the practice of algorithmic craft accessible for novice programers. Through it, I describe the design of each tool and discuss my experiences in engaging people in the creation of objects that are designed with programming, formed by machines, and shaped by hand. 
% \end{abstractpage}

\cleardoublepage

\section*{Acknowledgments}

To come....

%%%%%%%%%%%%%%%%%%%%%%%%%%%%%%%%%%%%%%%%%%%%%%%%%%%%%%%%%%%%%%%%%%%%%%
% -*-latex-*-
