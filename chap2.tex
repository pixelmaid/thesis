%% This is an example first chapter.  You should put chapter/appendix that you
%% write into a separate file, and add a line \include{yourfilename} to
%% main.tex, where `yourfilename.tex' is the name of the chapter/appendix file.
%% You can process specific files by typing their names in at the 
%% \files=
%% prompt when you run the file main.tex through LaTeX.
\chapter{Motivation and Background}

\textit{``The mathematician�s patterns, like the painter�s or the poet�s 
must be beautiful; the ideas like the colours or the words, must fit 
together in a harmonious way. Beauty is the first test: there is no 
permanent place in the world for ugly mathematics."
}

\begin{flushright}
G. H. Hardy \cite{hardy}
\end{flushright}
\section{Computational Design}\label{sec:computational_design}
The practice of computational design is fundamentally different from contemporary design. Due to the multitudes of approaches among different designers, it ultimately futile to attempt to describe a single standard design process. It is however useful to point out several key features of computational design that stand in contrast with the conventional design.

Both computational and conventional designers begin with a design problem. Conventional designers often proceed by roughing out a number of specific solutions to this problem. These early solutions are evaluated against one another for their successes and drawbacks, and from this evaluation a smaller set of more refined solutions are produced. This iterative process may continue, often through the incorporation of outside feedback, until a single solution is reached that is sufficiently refined and successful in addressing the initial problem. This highly simplified summary approximately describes the conventional iterative design process. Computational design incorporates many of the iterative principles of conventional design, but differs significantly in its approach. Rather than begin by developing a concrete initial solution to the initial design problem problem, the computational designer must first formalize the elements of the  problem into a set of rules. The designer then creates a system based on these rules that capable of producing a variety of solutions, depending on the input criteria it is given. In its simplest form, this system may consist of a single algorithm with static input and limited output  solutions. More frequently however, the computational design process produces complex systems that act on upon a wide range of input criteria and parameters, and can produce nearly infinite number of design solutions.
 Iteration in computational design then takes the form of incremental adjustments to the system. Naturally, many of the solutions produced by an initial system fail to address original design problem. By sampling a number of outputs from a system, the designer can "tweak" or make adjustments to the rules that govern the system, eventually resulting in more and more desirable output solutions. This process of sampling and tweaking is continued until the designer is satisfied with a given range of outputs. The designer can then vary the input to the system and can select among the resulting solutions. 
 
 While the process of computational design can be distinguished from conventional design, the two fields are compatible with one another. Aside from a wholly different approach, computational design can also be considered as a means of extending traditional design practice through several key affordances: These include the following:
\begin{itemize}
\item \textbf{Precision:} Computation supports high levels of numerical precision with relatively little effort on the part of the designer.
\item \textbf{Automation:}  Computation allows for rapid automation of repetitive tasks. Automation often plays a key role in enabling the development and transformation of complex patterns and structures, through the combination of large numbers of simple elements in an ordered and structured manner.
\item \textbf{Generativity and randomness:} Computation allows for the programmer to create algorithms which when run, allow for the computer to autonomously produce unique and often unexpected designs.
\item  \textbf{Parameterization:} Computation allows users to specify a set of degrees of freedom and constraints of a model and then adjust the values of the degrees of freedom while maintaining the constraints of the original model \cite{reas}.
\item \textbf{Documentation and remixing:} Computationally generated designs are generated by a program, which can be shared with and modified by other designers. Because these programs are often text-based, they also serve as a form of documentation of the design process. 
\end{itemize}	

In combination with these affordances however, computational design also incorporates  a number of challenges in the design process that are not present in traditional design:
\begin{itemize}
\item \textbf{Formalizing complex problems} As design problems grow in complexity, formalizing the problem in a manner that can be expressed programmatically becomes increasingly challenging. Writing an algorithm to generate a visual pattern is relatively simple, however writing a program to incorporate that pattern into the design of an entire garment  is non-trivial. 
\item \textbf{Creating singularities:} A designer will often choose to deviate from a set pattern or structure at specific points in order to create a special emphasis in that area. Because computational design is governed by a systematized ruleset, the methods of breaking these rules at arbitrary points is are often unclear and tedious to implement. 
\item \textbf{Selecting a final design:}  The systematic approach to computational design gives the designer the ability to produce extremely large numbers of solutions to a single design problem. While this is useful in situations where multiple solutions are required, when a single design must be chosen, the process of deciding on a solution is often difficult and sometimes arbitrary, especially if the decision is based on aesthetic criteria.
\end{itemize}

\section{Digital Fabrication} \todo{expand section}Although computational design must be conducted on a computer to some degree, the artifacts generated by computational design are not restricted to the screen. Digital fabrication  technology provides the opportunity to translate programmatically generated designs to physical form. Digital Fabrication is the process of using computer-controlled machines to fabricate objects specified by a digital design file or tool path. The machines that encompass digital fabrication range from  3D printers, laser cutters, and computer numerically controlled (CNC) milling machines, to vinyl cutters, CNC embroidery machines and knitting machines, and even inkjet printers.  Digital fabrication shares many of the affordances of computational design. In particular, it allows for the creation of physical objects of great complexity without formal skill in craft or extensive manual labor. Digital fabrication also allows for the rapid production of small volumes of similar or identical objects. Lastly, because the artifacts produced through digital fabrication are derived from digital files, anyone with access to the file, and a similar fabrication machine can create a copy of the object, or incorporate elements of it into a new design. 

Digital fabrication is also compelling for its own reasons. Currently, digital fabrication machines are rapidly decreasing in price and increasing in availability \cite{gershenfeld}. As a result, we are seeing the emergence of personal fabrication, wherein sophisticated manufacturing technologies are becoming available to regular people \cite{lipson}.  Excluding personal 3D printers which are generally limited to a few varieties of  ABS plastic, most personal fabrication machines can work with a wide range of materials. Laser cutters work well with traditional materials such as  wood, paper and cloth. Vinyl cutters can also be used on cloth and paper, as well as cut vinyl patterns which can be used for screen printing. 
The current stage of personal fabrication is estimated to be at the same place as personal computing in the 1970s \cite{eisenberg_fablearn}
\todo{add section on computer aided design}

\section{Algorithmic Craft}
The conjunction of computational design and digital fabrication has the potential to allow individuals to use programming to express their aesthetic concerns in the creation of objects. This is important because aesthetic expression through design is a substantial part of intellectual development [4] and an important part of people�s lives. These machines offer the potential to extend and innovate traditional forms of design, constructing and crafting by allowing for greater levels of automated complexity and precision in physical objects, and correspond well with the practice of programming. 
\todo{importance of materials- craft as domain of materiality, design and programing as abstractions in many cases, but deal very directly with materiality when applied to the real world- craft offers direct connection to this - citation from rosner article craft vs design}

\section{Challenges in broad participation in computational design and digital fabrication}
Despite the opportunity for casual, non-professional engagement in computational design and digital fabrication, this domain is largely limited to experts and professionals for a number of reasons.  In a practical context, new practitioners in this field are confronted with the difficult process of translating their code-based design to a format that is compatible with the target fabrication machine. Furthermore, the challenges involved in designing complex objects from multiple digitally fabricated parts are extremely difficult to tackle for casual users. There are also severe limitations on computational design software for novices capable of supporting digital fabrication. As we discuss in the following related work section, the majority of traditional CAD tools do not contain computational design capablities that are accessible to nicest users. Similarly, novice oriented programing environments lack the functionality to allow novices to produce designs that are suitable for fabrication. More broadly, there are significant perceptual barriers to participation. There persists among the general public a  limited perception of the applications of programing. Many people consider programing to be irrelevant to their interests, and therefore lack motivation to pursue what they perceive to be a highly specialized and difficult undertaking [12]. There are also prevailing perceptions of digital fabrication which may hinder casual engagement. Personal fabrication technology is often portrayed as a precursor to the production of replicator-like technology which can instantiate literally anything by building it directly from atoms. This projection of future technology is exciting to think about, but I argue that it also acts as a barrier to immediate widespread engagement with existing forms of digital fabrication, by setting up unreal expectations for this technology and portraying it as technology that facilitates new forms of consumerism, as opposed to being a new tool for personal creation and expression. This perspective also eliminates the need or desire for human engagement in the fabrication process, eliminating the entry points for craft: 

 \textit{A central element of these and  other visions of the future is that  craft is done for us: Kitchens tell us what and how to cook, eliminating the creativity and pleasure of  cooking from scratch with what�s on hand; object printers create flawless prototypes, eliminating messily glued-together chipboard and toothpicks. In this new 
world, craft becomes fetish�the proudly displayed collection of vinyl records shelved alongside an iPod and digital files \cite{rosner_craft_vs_design}.}

There is also the tendency to trivialize the hobbyist applications of digital fabrication when analyzed in a research context. In the domain of Human Computer Interaction (HCI), researchers often focus on the hedonistic properties technologically oriented DIY practices as opposed to the utility  of the resultant artifacts or their ability to generate profit. Pleasure and self-expression are central components of hobbyist  and craft-oriented computation and digital fabrication, however these qualities do not come at the cost of generating artifacts that are practical, functional, and sellable \cite{tanenbaum}. The trend of separating hobbyist practice as merely fun in contrast to professional practical applications overshadows some of the most interesting practical possibilities that emerge through amateur use of this technology. 


